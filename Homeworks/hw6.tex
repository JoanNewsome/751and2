\documentclass[12pt]{article}
\usepackage{geometry,amsmath,amssymb, graphicx, natbib, float, enumerate}
\geometry{margin=1in}
\renewcommand{\familydefault}{cmss}
\restylefloat{table}
\restylefloat{figure}

\newcommand{\code}[1]{\texttt{#1}}
\newcommand{\Var}{\mathrm{Var}}
\newcommand{\logit}{\mathrm{logit}}

\begin{document}
\noindent
{\bf BST 140.752 \\ Problem Set 5} \\
\section{LMM}
\begin{enumerate}
\item Let $Y_{ij} = \mu + u_i + \epsilon_{ij}$ for $u_i \sim N(0,\sigma^2_u)$ and $\epsilon_{ij} \sim N(0, \sigma^2)$. Calculate the BLUP 
for $u_i$
\item Let $Y = X \beta + Z u + \epsilon$ for $u \sim N(0,\Sigma_u)$ and $\epsilon \sim N(0, \sigma^2 I)$. Calculate the BLUP for $u$. 
\item Load the Rail data set in R. Fit a mixed model of the form from question 1. Compare a the estimates of the mean for each rail with the empirical mean.	
\item Load the pixel data set in R. Fit a linear mixed effect model where you have $Y_{ijk} = \beta_0 + \beta_1 x_{k} + u_{i} + u_{ij} + \epsilon_{ijk}$
where $Y_{ijk}$ is pixel, $i$ is dog, $j$ is side and $k$ is day index and $x_k$ is day. Fit the model and interpret the results.
\item Consider the model $Y_{i} = \mu + \epsilon_{ij}$. Consider putting a so-called "flat" prior on $\mu$. That is acting like a distribution
that is $1$ from $-\infty$ to $+\infty$ is a valid density. Calculate the distribution marginalized over $\mu$ and show that it is the same likelihood
used to obtain the REML estimates.
\end{enumerate}

\section{FFT}
\begin{enumerate}
\item Consider a time series $Y_t$ for $t=1,\ldots,n$. Give the correspondence between the terms of the discrete Fourier transform of the data and the
fit using linear models with a trigonometric basis.
\item Write an R program that takes in a chord (the addition of three notes) and guesses the notes. The frequencies for the various notes can be found here
\texttt{http://www.phy.mtu.edu/~suits/notefreqs.html}
\item Consider the Haar wavelet basis for a time series $Y_t$ for $t=1\,ldots, 2^k$. Derive the correspondence
between the wavelet coefficients and binned means of the time series.
\end{enumerate}

\section{Principal components}
\begin{enumerate}
\item Formally prove that the population principal components explain the maximum variability subject to being linear combinations of the data
and orthogonal to the others.
\end{enumerate}

\end{document}
