\documentclass[12pt]{article}
\usepackage{geometry,amsmath,amssymb, graphicx, natbib, float, enumerate}
\geometry{margin=1in}
\renewcommand{\familydefault}{cmss}
\restylefloat{table}
\restylefloat{figure}

\newcommand{\code}[1]{\texttt{#1}}
\newcommand{\Var}{\mathrm{Var}}
\newcommand{\logit}{\mathrm{logit}}

\begin{document}
\noindent
{\bf BST 140.752 \\ Problem Set 3} \\
\section{Hypothesis testing}
\begin{enumerate}
\item A special study is conducted to test the hypothesis that
  persons with glaucoma have higher blood pressure than average. Two
  hundred subjects with glaucoma are recruited with a sample mean
  systolic blood pressure of $140mm$ and a sample standard deviation
  of $25mm$. (Do not use a computer for this problem.)
  \begin{enumerate}[a.]
  \item Construct a $95\%$ confidence interval for the mean systolic
    blood pressure among persons with glaucoma. Do you need to
    assume normality? Explain.
  \item If the average systolic blood pressure for persons without
    glaucoma of comparable age is $130mm$. Is there statistical
    evidence that the blood pressure is elevated? Perform the relevant
    test and report a P-value.
  \end{enumerate}
\item  Consider the previous question. 
  \begin{enumerate}[a.]
  \item Make a probabilistic argument that the interval
    $$\left[\bar X - z_{.95} \frac{s}{\sqrt{n}}, ~~\infty\right]$$
    is a 95\% {\em lower bound} for $\mu$. 
  \end{enumerate}
\item  Suppose we wish to estimate the concentration $\mu$g/m$\ell$
  of a specific dose of ampicillin in the urine.  We recruit 25
  volunteers and find that they have sample mean concentration of
  7.0 $\mu$g/m\ $\ell$ with sample standard deviation 3.0 $\mu$
  g/m$\ell$.  Let us assume that the underlying population
  distribution of concentrations is normally distributed.
  \begin{enumerate}[a.]
  \item Find a 90\% confidence interval for the population mean concentration.
  \item How large a sample would be needed to insure that the length
    of the confidence interval is 0.5 $\mu$ g/m$\ell$ if it is assumed
    that the sample standard deviation remains at 3.0 $\mu$ g/m\ $\ell$?
  \end{enumerate}
\item A study of blood alcohol levels (mg/100 ml) at post mortem
  examination from traffic accident victims involved taking one blood
  sample from the leg, A, and another from the heart, B.  The results
  were:
\begin{center}
\ttfamily
\begin{tabular}{lrrrrr} 
\hline
Case & A & B & Case & A & B   \\ \hline
1 &  44 &   44 &  11 & 265 &  277 \\
2 & 265 &  269 &  12 &  27 &   39 \\
3 & 250 &  256 &  13 &  68 &   84 \\
4 & 153 &  154 &  14 & 230 &  228 \\
5 &  88 &  83  &  15 & 180 &  187 \\
6 & 180 &  185 &  16 & 149 &  155 \\
7 &  35 &   36 &  17 & 286 &  290 \\
8 & 494 &  502 &  18 &  72 &  80  \\
9 & 249 &  249 &  19 &  39 &  50  \\
10& 204 &  208 &  20 & 272 &  271 \\ \hline
\end{tabular}
\end{center}
\normalfont Test whether or not the mean blood alcohol level differs
between the heart and the leg.  Give the appropriate null and
alternative hypotheses. Give the relevant P-value. Interpret your
results, state your assumptions.
\item Forced expiratory volume FEV is a standard measure of pulmonary
  function.  We would expect that any reasonable measure of pulmonary
  function would reflect the fact that a person's pulmonary function
  declines with age after age 20.  Suppose we test this hypothesis by
  looking at 10 nonsmoking males ages 35-39, heights 68-72 inches and
  measure their FEV initially and then once again 2 years later.  We
  obtain this data.
\begin{center}
\begin{tabular}{cccccc}
\hline
& Year 0 & Year 2 && Year 0 & Year 2 \\
& FEV & FEV && FEV & FEV \\
Person & (L) & (L) & Person & (L) & (L) \vspace{+0.05in} \\ \hline
1 & 3.22 & 2.95 & 6 & 3.25 & 3.20 \\
2 & 4.06 & 3.75 & 7 & 4.20 & 3.90 \\
3 & 3.85 & 4.00 & 8 & 3.05 & 2.76 \\
4 & 3.50 & 3.42 & 9 & 2.86 & 2.75 \\
5 & 2.80 & 2.77 & 10 & 3.50 & 3.32 \vspace{+0.05in} \\ \hline
\end{tabular}
\end{center}
\begin{enumerate}[a.]
\item Preform and interpret the relevant test.  Give
  the appropriate null and alternative hypotheses. Intepret your
  results, state your assumptions and give a P-value.
\end{enumerate}
\item Another aspect of the preceding study involves looking at the
  effect of smoking on baseline pulmonary function and on change in
  pulmonary function over time.  We must be careful since FEV depends
  on many factors, particularly age and height.  Suppose we have a
  comparable group of 15 men in the same age and height group who are
  smokers and we measure their FEV at year 0.  The data are given (For
  purposes of this exercise assume equal variance where appropriate).
\begin{center}
\begin{tabular}{cccccc} \hline
& FEV & FEV && FEV & FEV \\
& Year 0 & Year 2 && Year 0 & Year 2 \\
Person & (L) & (L) & Person & (L) & (L) \vspace{+0.05in} \\ \hline
1 & 2.85 & 2.88 & 9 & 2.76 & 3.02 \\
2 & 3.32 & 3.40 & 10 & 3.00 & 3.08 \\
3 & 3.01 & 3.02 & 11 & 3.26 & 3.00 \\
4 & 2.95 & 2.84 & 12 & 2.84 & 3.40 \\
5 & 2.78 & 2.75 & 13 & 2.50 & 2.59 \\
6 & 2.86 & 3.20 & 14 & 3.59 & 3.29 \\
7 & 2.78 & 2.96 & 15 & 3.30 & \ 3.32 \\
8 & 2.90 & 2.74 &&& \vspace{+0.05in} \\ \hline
\end{tabular}
\end{center}
Test the hypothesis that the change in FEV is equivalent between
non-smokers and smokers. State relevant assumptions and interpret
your result. Give the relevant P-value.

\item  Suppose that systolic blood pressures were taken on $16$
oral contraceptive users and $16$ controls at baseline and again then
two years later. The average difference from follow-up SBP to the
baseline (followup - baseline) was $11$ $mmHg$ for oral contraceptive
users and $4$ $mmHg$ for controls.  The corresponding standard
deviations of the differences was $20$ $mmHg$ for OC users and $28$
$mmHg$ for controls.
\begin{enumerate}[a.]
\item Calculate and interpret a $95\%$ confidence interval for the
  {\bf relative} change in systolic blood pressure for oral
  contraceptive users; assume normality on the log scale.
\item Does the change in SBP over the two year period appear to differ
  between oral contraceptive users and controls? Perform the relevant
  hypothesis test and interpret. Give a P-value. Assume normality and
  a common variance.
\end{enumerate}  
\end{enumerate}

\section{Inference and estimation in linear models}
\begin{enumerate}[1.]
\item Consider the linear model $Y = X\beta + \epsilon$ and $\epsilon \sim N(0, \sigma^2 I)$. Do the following
\begin{enumerate}[A.]
\item Derive the ML estimate, $\hat \beta$.
\item Derive the variance of the ML estimate.
\item Show that $\hat \beta$ is independent of the residual vector, $e$.
\end{enumerate}
\item Let $\beta_j$ be an element of $\beta$ from the previous problem. Let $\hat SE_{\hat \beta_j}$ be the standard
	error of $\hat \beta_j$, the ML estimate of $\beta_j$. Argue that $(\hat \beta_j - \beta_j) / \hat SE_{\hat \beta_j}$
	follows a T distribution with $n-p$ degrees of freedom. Use this to create a confidence interval for $\hat \beta_j$.
\item Let $Y_{ij} = \mu_i + \epsilon_{ij}$ for $i=1,2$ and $j=1,\ldots,J_i$ where the 
	$\epsilon_{ij} \sim N(0, \sigma^2)$ are iid.
	 Show that the unbiased estimate of $\sigma^2$ is the so-called pooled variance estimate, $S_p^2 = 
	\frac{1}{J_1 + J_2 - 2}\{(J_1 - 1) S_1^2 + (J_2 - 1) S_2^2\}$ where $S_i^2$ is the standard variance estimate
	within group $i$. Derive a $T$ confidence interval for $\mu_1 - \mu_2$ and test of $\mu_1 = \mu_2$.
\item Derive the variance estimate from the previous problem of $i=1,\ldots I$. Derive an overall $F$ test
	for the hypothesis that $H_0:\mu_1= \mu_2 = \ldots=\mu_I$ versus the alternative that at least two
	are unequal. Argue that this F test compares the variation between the groups to that within the groups. (This
 	is called the general ANOVA F test.)
\item Let $Y = X \beta + \epsilon$ where $\epsilon \sim N(0,\Sigma)$. 
	\begin{enumerate}[A.]
	\item Argue that for any $W$, including $I$, $\hat \beta (W) = (X'WX)^{-1}X'WY$ is an unbiased estimate of $\beta$.
	\item What is the variance of $\hat \beta (W)$?
	\item Argue that $\hat \beta (\Sigma^{-1})$ is the ML estimate if $\Sigma$ were known.
	\item Use the previous result to calculate the MLE of $\mu = (\mu_1,\ldots,\mu_I)'$ when
		$Y_{ij} = \mu_i + \epsilon_{ij}$ where the $\epsilon_{ij}$ are independent Gaussians with mean $0$ and
		variance $\sigma^2_i$. 
	\end{enumerate}
\item Consider the linear regression model $Y_i = \beta_0 + \beta_1 X_i + \epsilon_i$. Argue that the variance of
	$\hat \beta_1$ is minimized with the variance in the observed $X_i$ is maximized. Ergo, the lowest variance
	estimate is obtained with what pattern in the $X_i$?	
\item Consider the general linear hypothesis $H_0: K\beta = m$ versus $H_a : K \beta \neq m$. Go through a careful development of the general F test.
\item Derive the T confidence interval, T-test and F test associated with the hypothesis $H_0: \beta_k = 0$ versus $H_a : \beta_k \neq 0$ where $\beta_k$ is a component of $\beta$.
\item Give a proof that if $X = [X_1 X_2]$ with $X_1$ as $1 \times n$ and $X_2$ is $(p-1)\times n$ where $Y = X \beta + \epsilon$ and $\beta = [\beta_1 \beta_2]'$ with $\beta_1$ as $1\times 1$ then $\hat \beta_1 = e_{y|X_1} e_{X_1 | X_2} / <e_{X_1 | X_2}, e_{X_1 | X_2}>$ where $e_{A|B} = (I - B (B'B)^{-1} B') A$.
\item Show that the $T$ confidence interval for $\beta_1 - \beta_2$ for the model $Y_{ij} = \beta_i + \epsilon_{ij}$ for $i = 1, 2$ and $j = 1,\ldots, J_i$ and $\epsilon_{ij} \sim N(0,\sigma^2)$
is $\bar Y_1 - \bar Y_2 \pm t_{1-\alpha/2, J_1 + J_2 - 2} S_p \sqrt{\frac{1}{J_1} + \frac{1}{J_2}}$. (This is the standard interval given for two group differences in introductory statistics classes).

 
\end{enumerate}

\section{Coding and data analysis exercises}
\begin{enumerate}
\item Perform the following simulation. Randomly simulate $1,000$
  sample means of size $16$ from a normal distribution with means $5$
  and variances $1$. Calculate $1,000$ test statistics for a test of
  $H_0:\mu = 5$ versus $H_a:\mu< 5$. Using these test statistics calculate
  $1,000$ P-values for this test. Plot a histogram of the P-values. Note, this
  exercise demonstrates the fact that the distribution of P-values is
  uniform.
\item Here we will verify that standardized means of iid normal data
  follow Gossett's $t$ distribution. Randomly generate $1,000 \times
  20$ normals with mean $5$ and variance $2$. Place these results in a
  matrix with $1,000$ rows. Using two \texttt{apply} statements on the
  matrix, create two vectors, one of the sample mean from each row and
  one of the sample standard deviation from each row. From these
  $1,000$ means and standard deviations, create $1,000$ $t$
  statistics.  Now use R's \texttt{rt} function to directly generate
  $1,000$ $t$ random variables with $19$ df. Use R's \texttt{qqplot}
  function to plot the quantiles of the constructed $t$ random
  variables versus R's $t$ random variables. Do the quantiles agree?
  Describe why they should.
\item Here we will verify the chi-squared result. Simulate $1,000$
  sample variances of $20$ observations from a normal distribution
  with mean $5$ and variance $2$.  Convert these sample variances so
  that they should be chi-squared random variables with $19$ degrees
  of freedom. Now simulate $1,000$ random chi-squared variables with
  $19$ degrees of freedom using R's \texttt{rchisq} function. Use R's
  \texttt{qqplot} function to plot the quantiles of the constructed
  chi-squared random variables versus those of R's random chi-squared
  variables. Do the quantiles agree? Describe why they should.
\item Download the data at
\begin{verbatim}
http://dl.dropbox.com/u/95701/751.2/takeHomeData.zip
\end{verbatim}
The data documentation are as follows
The data are obtained from the Sleep Heart Health Study, though having been modified for the exercise and for data confidentiality, so that numbers from this data set will not match with published numbers from the same study.

\begin{verbatim}
Variables:
1.	tst � total sleep time in hours;
2.	events � number of sleep related events over the night
3.	meds � was the subject on anti-hypertensive medications (1=yes, 0=no).
4.	sbp � systolic blood pressure mmHg
5.	dbp � diastolic blood pressure mmHg
6.	age � age in years
7.	bmi � body mass index kg / m2
8.	race � (1=w 2=b 3=Nat Am/Alaskan 4=Asian/PI 5=hisp/Mex Amer 6=other)
9.	gender � (1 =male, 0=female)
10.	alcohol � (number of drinks per week)
11.	smoke � (ever smoked cigarettes, at least 20 packs in a lifetime, 1=yes, 0=no)
12.	Waist/Hip ratio
\end{verbatim}

Of interest is the rate of respiratory disturbances (�events�) per hours slept.  Per common practice in the field, the manuscripts referenced above refer to this rate as the apnea/hypopnea index (AHI) or respiratory disturbance index (RDI).  
\begin{enumerate}[A.]
\item Fit a linear model to consider the relationship between the Log(respiratory disturbance index + 1) (response) and BMI (predictor).
\item Create exploratory graphics to investigate the relationship.
\item Test the hypothesis that RDI is associated with BMI.
\item Create and interpret relevant confidence intervals.
\item Create relevant residual plots.
\item Investigate missing data patterns.
\end{enumerate}
Use the article ''Association of sleep disordered breathing, sleep apnea and hypertension in a large, community based study" as your guide for including confounders.

\item Download the data from
\begin{verbatim}
https://dl.dropboxusercontent.com/u/95701/teams.zip
\end{verbatim}
Some documentation for the data can be found at:
\begin{verbatim}
https://dl.dropboxusercontent.com/u/95701/teamsDocumentation.docx
\end{verbatim}
Answer the following
\begin{enumerate}[A.]
\item What team performance predictor variables are most useful for predicting winning percentage (games won divided by games played) among teams since 1970? 
\end{enumerate}
\item Write an R function, myLM. The function should take in a Y vector and X matrix and do the following:
\begin{enumerate}[A.]
\item Check to make sure that X and Y have the right format (matrix and vector respectively), have the right dimensions, have no missing, NA or Inf, are numeric and that X is full rank.
\item Return a list with the following information:
\begin{enumerate}
\item The least squares estimate of the associated linear model.
\item R squared.
\item A T table (estimate, standard error, t statistics, P-value).
\item A residual vector.
\item A vector of fitted values.
\item A vector hat diagonals.
\end{enumerate}
\item Test it out on models that you fit for the previous problem and make sure that your numbers agree with that of R's lm.
\end{enumerate}
\item Go to Leonard Stefanski's web page on residuals plots. Read the associated American Statistician article. Reproduce at the figures for the "Correct figure", "X marks the spot" and "Homer Simpson" examples using your myLM function.
\end{enumerate}


\end{document}
