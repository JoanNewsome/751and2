\documentclass[12pt]{article}
\usepackage{geometry,amsmath,amssymb, graphicx, natbib, float, enumerate}
\geometry{margin=1in}
\renewcommand{\familydefault}{cmss}
\restylefloat{table}
\restylefloat{figure}

\newcommand{\code}[1]{\texttt{#1}}
\newcommand{\Var}{\mathrm{Var}}
\newcommand{\logit}{\mathrm{logit}}

\begin{document}
\noindent
{\bf BST 140.751 \\ Problem Set 1} \\

\section{Least squares}



\section{Linear models}
\begin{enumerate}[1.]
\item Let $Y_i = \beta_0 + \beta_1 X_i + \epsilon_i$ where $\epsilon_i \sim N(0,\sigma^2)$ for $i=1,\ldots,n$.
  \begin{enumerate}[A.]
  \item Derive the MLEs for $\beta_0$, $\beta_1$ and $\sigma^2$.
  \item Relate $\beta_1$ to the correlation between $Y_i$ and $X_i$.
  \item Suppose that you standardize (i.e. take $(Y_i - \bar Y) / S_y$ and $(X_i - \bar X) / S_X$) $X_i$ and $Y_i$. Derive
    the estimates of $\beta_0$ and $\beta_1$.
  \end{enumerate}
\item Let $Y_{ij} = \alpha_0 + \beta_j + \epsilon_{ij}$ for $i = 1,\ldots, I$ and $j = 1,\ldots, J$.
  \begin{enumerate}[A.]
  \item Write out the design matrix for the associated linear model.
  \item Show what the estimates are under the following constraints:
    \begin{enumerate}
    \item $\alpha_0 = 0$
    \item $\beta_1 = 0$
    \item $\beta_J = 0$
    \item $\sum_{j=1}^J \beta_j = 0$
    \end{enumerate}
  \end{enumerate}
\item Let $\Sigma$ be a known matrix. Consider the model $Y = X\beta + \epsilon$ where $\epsilon \sim N(0, \Sigma)$. 
  Derive the ML estimate of $\beta$.
\item Let $P$ be a rotation matrix and consider the model $Y = X\beta
  + \epsilon$ where $\epsilon \sim N(0, \sigma^2 I)$.  Suppose someone gave you the
  ML estimates for $\tilde \beta$ and $\tilde \sigma^2$ from fitting
  the model $\tilde Y = \tilde X \tilde \beta + \tilde \epsilon$ where
  $\tilde Y = P Y$ and $\tilde X = PX$ and $\tilde \epsilon \sim N(0,
  \tilde \sigma^2)$. Relate these estimates to the ML estimates
  of $\beta$ and $\sigma^2$.
\item Let $Y ~|~ \beta \sim N(X\beta, \sigma^2 I)$ and $\beta \sim N(\beta0, \tau^2 I)$. What is the posterior distribution of $\beta$?
\item Consider the model $Y = X\beta + \epsilon$. Let $F$ be an invertible $p\times p$ matrix and $\tilde X = X F$. 
\begin{enumerate}[A.]
	\item Consider another model $Y = \tilde X \tilde \beta + \epsilon$. Argue that the models are equivalent.
	\item Show that the least squares estimate of $\tilde \beta$ from the second model is $F^{-1} \hat \beta$ where $\hat \beta$ is the least squares
	estimate from the first model.
	\item Suppose that you have a linear regression equation where one of the regressors is temperature. Use the results above to relate the
		beta coefficients if the regressor is input as Celsius or Fahrenheit.
\end{enumerate}
\item Consider a linear model with iid errors $N(0, \sigma^2)$ errors. Show that $\frac{1}{n-p} e'e$, where $e$ is the vector of residuals, is the
	ML estimate of $\sigma^2$. Further show that this estimate is unbiased.
\begin{enumerate}[A.]
	\item Argue that $\frac{1}{\sigma^2} (y - X\beta)' (y - X\beta)$ is $\chi^2_n$
	\item Argue that $\frac{1}{\sigma^2} e'e$ is $\chi^2_{n-p}$.
	\item Argue that $\frac{1}{\sigma^2} (y - X \beta)' X(X'X)^{-1} X' (y - X\beta)$ is $\chi^2_p$.
	\item In each of the above cases, use the expected value calculation for quadratic forms to verify that the expected values equals the Chi squared df.
\end{enumerate}
\end{enumerate}


\section{Multivariate means, variances and normals}
\begin{enumerate}[1.]
\item Let $X$ be a multivariate vector with mean $\mu$. Show that $E[AX + b] = A\mu + b$.
\item Consider the previous problem; assume that $\mbox{Var}(X) = \Sigma$. Show that $\mbox{Var}(AX + b) = A\Sigma A'$.
\item Show that $E[(X - \mu)(X - \mu)'] = E[XX'] - \mu\mu'$.
\item Argue that $\mbox{Var}(X)$ is non-negative definite.
\item Let $C(X, Y)$ be the multivariate covariance function, $E[(X - \mu_x) (Y - \mu_y)']$.
  Show that $C(X, Y) = E[XY'] - \mu_X\mu_y'$. 
\item Show that $C(X_1 + X_2, Y) = C(X_1, Y) + C(X_2, Y)$.
\item Argue that $C(X, Y) = C(Y, X)'$. 
\item Argue that $Var(X + Y) = Var(X) + C(X, Y) + C(Y, X) + V(Y)$.
\item Argue that $C(AX, BY) = AC(X, Y)B'$.  
\item Let $X \sim N(0, I)$. Argue that $a X / \sqrt{a'a} \sim N(0, 1)$ for any non-zero vector $a$. 
\item Let $X \sim N(0, I)$. Argue that if $AA' = I$ then $AX \sim N(0, I)$. Argue geometrically why this occurs.
\item Let $X_i$ for $i=1,\ldots, I$ be iid $k$ dimensional vectors
from a distribution with mean $\mu$ and variance $\Sigma$. 
What is the mean and variance of the multivariate pointwise sample
average of the vectors?
\item Let $X_i$ be iid $k$ dimensional vectors
from a distribution with mean $\mu$ and variance $\Sigma$.
Give an unbiased estimate of $\Sigma$ when $\mu$ is known.
\item Consider a covariance matrix that is of the form
$$
\sigma^2 \mathbf{I} + \theta \mathbf{1}\mathbf{1}'
$$	where $\sigma^2$ and $\theta$ are positive constants
and $\mathbf{1}$ is a vector of ones. Argue that this
matrix describes random vectors where every pair of elements of 
the vector are 
equally correlated and every elemant has the same variance. Give
this correlation and variance.
\item Let $X = (X_1' ~ X_2')' \sim N(\mu, \Sigma)$.
  \begin{enumerate}[A.]
  \item Derive the marginal distribution of $X_1$
  \item Derive the conditional distribution $X_1 ~|~ X_2$. 
  \end{enumerate}
\item Let $X ~|~ \mu \sim N(\mu, \Sigma)$ and $\mu \sim N(\alpha, \tau I)$. Derive the distribution of
  $\mu ~|~ X$.
 \item Argue that if $Y \sim N(\mu, \Sigma)$, the quadratic form $(Y - \mu)'\Sigma^{-1}(Y - \mu)$ is $\chi^2_p$. 
\end{enumerate}



\section{Computing and analysis}
\begin{enumerate}[1.]
\item Write an R function that takes a $Y$ vector and $X$ matrix and obtains
the least squares fit for the associated linear model.
\item Write and R function that takes an $n\times p$ data matrix, $X$, and
	``whitens'' it via subtracting out a mean and multiplying by a matrix
	so that the resulting matrix has ($p$) sample column means of 0 and
	$p\times p$ sample covariance matrix of $I$.
\item You collect 100 blood pressure measurements from a population. Twenty one exhibit high blood pressure. 
  Using a beta prior, plot the posteriors assuming: Beta(1, 1), Beta(2, 2), Beta(.5, .5). Calculate 95\% 
  equi-tail and HPD credible intervals.
\item You now collect 100 measurements from an otherwise similar population with different diets; 15 have high blood pressure.
  Plot the posteriors for the relative risk, risk difference and risk ratio.  Assume Beta(1,1) priors for both populations.
\item One nuclear reactor test failed $51$ times after having been monitored for $1,000$ days. Assuming a Gamma(.1,.1) prior
and a Poisson model, plot the posterior for the failure rate. Give a credible interval (HPD and equi-tail). 
\item A second reactor failed $25$ times for $600$ monitoring
  days. Plot the posterior and give credible intervals (HPD and equi-tail) for the
  relative rate with the other reactor.
\end{enumerate}

\end{document}
