\documentclass[12pt]{article}
\usepackage{geometry,amsmath,amssymb, graphicx, natbib, float, enumerate}
\geometry{margin=1in}
\renewcommand{\familydefault}{cmss}
\restylefloat{table}
\restylefloat{figure}

\newcommand{\code}[1]{\texttt{#1}}
\newcommand{\Var}{\mathrm{Var}}
\newcommand{\logit}{\mathrm{logit}}

\begin{document}
\noindent
{\bf BST 140.752 \\ Problem Set 4} \\
\section{Residuals}
\begin{enumerate}
\item Consider a linear model $Y = X\beta + \delta \Delta + \epsilon$ where $\delta$ is a vector with a $1$ at position $i_0$ and $0$ elsewhere. Argue the following.
\begin{enumerate}[A.]
\item The $i_0$ residual is $0$ for this model.
\item The fitted value for $\beta$ using all of the data and this model is equivalent to that using only the data with the $i_0$ observation deleted.
\item Argue that the standardized Press residuals are a test statistic for $\Delta = 0$.
\end{enumerate}
\item Consider the residuals for the ordinary linear model. Derive their mean and variance. 
\item Carefully write up the proof that relates the Press residuals to the ordinary residuals. Derive the mean and variance/covariance of the Press residuals.
\item Prove the Sherman/Morrison/Woodburry theorem.
\item Prove that the hat matrix diagonals are between 0 and 1.
\item Why are the studentized residuals not exactly distributed as $t$ statistics?
\end{enumerate}

\section{Inference under incorrectly specified models}
For all of this section, let Model 1 be $Y = X_1 \beta_1 + \epsilon$ and Model 2 be $Y = X_1 \beta_1 + X_2 \beta_2 + \tilde \epsilon$. 
\begin{enumerate}
\item Suppose that Model 1 is fit while Model 2 represents the actual truth. Give the bias and variance of $\beta_1$. Give the
expected value of $S^2$.
\item Suppose that Model 2 is fit while Model 1 is true. Give the bias and variance of the estimted $\beta$. Give the expected value of $S^2$. 
\end{enumerate}


\end{enumerate}

\end{document}


\section{GLMs}
\begin{enumerate}
\item Calculate the mean and variance of a random variable from an exponential family using the cumulant generating function.
\item Show that when using a canonical link function, the Fisher
  scoring and Newton Raphson algorithms for finding glm MLEs
  are identical.
\item Show that, using the notation from class, for known $\phi$ and a
  canonical link, the sufficient statistics for a glm are $X^t y$ where
  $t$ denotes a transpose.
\item Suppose that $y_i$ is Poisson with $g(\mu_i) = \alpha + \beta x_i$
  where $g$ is the link function and $x_i=1$ for $i=1,\ldots,n_a$ and
  $x_i=0$ for $i= n_a + 1, \ldots, n_a + n_b$. That is, $x_i$, is a treatment
  indicator for two groups, $A$ and $B$. Show that, regardless of the link
  function, the fitted means equal the two sample means.
 \item Consider the class of {\em binary} glms where the link function
  satisfies $g\{\mu(x)\} = \Phi^{-1}\{\mu(x)\} = \alpha + \beta x$
  where $\Phi(\cdot)$ is a distribution function and $\mu(x)$ is the
  Bernoulli mean. Let $\phi$ be the (assumed continuous) associated
  density. Show that the $x$ at which $\mu(x) = .5$ is
  $x=-\alpha/\beta$.  Further show that the rate of change of $\mu(x)$
  at this point is $\beta \phi(0)$. Illustrate that this is $.25\beta$
  for the logit link and $\beta / \sqrt{2\pi}$ for the probit link.
\end{enumerate}  

\section{Coding and data analysis exercises}
\begin{enumerate}
\item Consider the sleep data from the previous homework.
\begin{enumerate}[A.]
\item Consider the model fit from the previous homework. Write a program to grab the hat diagnals as well as use R's lm to obtain them directly. 
	Look at the influence of various data points.
\item Consider the model fit from the previous homework. Write a program to grab the residuals and Press residuals. Investigate these residuals
	in the context of this model.
\end{enumerate}
\item Consider the baseball data from the previous exercise.
\begin{enumerate}[A.]
\item Consider the model fit from the previous homework. Write a program to grab the hat diagnals as well as use R's lm to obtain them directly. 
	Look at the influence of various data points.
\item Consider the model fit from the previous homework. Write a program to grab the residuals and Press residuals. Investigate these residuals
	in the context of this model.
\end{enumerate}
\item Write a function that takes a $Y$ ($n\times 1$) an $X_1$ ($n \times 1$) and an $X_2$ ($n \times (p-1)$) and produces the partial regression
	plot of $e_{Y | X_2}$ by $e_{Y | X_2}$. 
\item Consider the Challenger O-ring data
\item The table below shows the temperature (Temp in Fahrenheit) and presence (1) or
  absence of O-ring distress (OD) at the time of flight for the 23 flights
  before the 1986 Challenger mission disaster.
\begin{center}
  \begin{tabular}{cc|cc|cc|cc|cc}
    \hline
    Temp & OD & Temp & OD & Temp & OD & Temp & OD & Temp & OD  \\ \hline
    66 & 0 & 70 & 1 & 69 & 0 & 68 & 0 & 67 & 0 \\
    72 & 0 & 73 & 0 & 70 & 0 & 57 & 1 & 63 & 1 \\
    70 & 1 & 78 & 0 & 67 & 0 & 53 & 1 & 67 & 0 \\
    75 & 0 & 70 & 0 & 81 & 0 & 76 & 0 & 79 & 0 \\
    75 & 1 & 76 & 0 & 58 & 1 &    &   &    &   \\ \hline
  \end{tabular}
\end{center}
\begin{enumerate}[A.]
\item Use logistic regression to model the effect of temperature on the probability
  of thermal distress. Interpret the results. Plot a figure of the fitted model.
\item Estimate the probability of thermal distress at 31 degrees, which was the
  temperature at the time of the Challenger flight.
\item Construct a profile likelihood for the effect of temperature on the odds
  of thermal distress, interpret.
\item Check model fit by comparing this model to a more complex model.
\end{enumerate}